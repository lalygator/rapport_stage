\begin{abstract}
%L'enjeu
% La question de notre solitude dans l'univers s'est posé tout au long de l'existence de l'humanité, 
\vspace{-1.9em}
\item \paragraph{Contexte}
La recherche de mondes habitables étant l’un des grands enjeux astronomiques de notre décennie, il est nécessaire de pouvoir développer des outils permettant l’observation et la caractérisation d’exoplanètes de types terrestre. Ces dernières étant généralement situées très proches de leur étoile (de l’ordre de $a=1\text{ UA}$), distinguer la lumière de la planète de celle de son étoile s’avère particulièrement complexe, de par la diffraction de la lumière dans système optique et les influences instrumentales provoquant une perte de résolution dans l'image.

\vspace{-1.9em}
\item \paragraph{Objectif} 
Proxima b, l'exoplanète la plus proche de notre système solaire, est une cible de choix pour cette étude. \textsl{Mettre une phrase de justification supplémentaire ?} On souhaite ainsi chercher un outil nous permettant de récupérer suffisamment sa lumière pour permettre son analyse. Pour cela, nous nous appuierons sur \textbf{l’apodisation}, et notamment l'utilisation de masques pupilles pour déterminer sa forme.

\vspace{-1.9em}
\item \paragraph{Méthodes}
À l'aide de l'apodisation, nous chercherons comment modifier le signal lumineux de l'étoile afin de créer une zone de haut contraste permettant de récupérer correctement la lumière de la planète. On pourra ainsi analyser la lumière réfléchie de cette planète et en déduire sa composition atmosphérique, permettant ainsi la potentielle détection de biosignatures.

\vspace{-1.9em}
\item \paragraph{Résultats}
On trouve que l'apodiseur le plus optimal pour l'observation de Proxima b est un apodiseur de paramètres IWA = 2.6, OWA = 6.5 et T = 0.70. Ce masque pupille permet ainsi de réduire le temps de télescope d'un facteur 8, bien qu'on reste toujours limité par le flou d'optique adaptative. %Si on pouvait réduire ce flou jusqu'à $10^{??}$, alors on pourrait espérer augmenter le gain de SNR jusqu'à ???

\vspace{-1.9em}
\item \paragraph{Conclusion}
Bien que nous puissons atteindre des niveaux de contraste très élevé avec un apodiseur, nous serons toujours ultimement limité par le flou d'optique adaptative. De plus, l'observation directe de Proxima b n'ayant pas encore été effectué, nous n'avons pas la certitude que l'apodiseur proposé soit le plus optimal pour cette observation. Il faudra donc réaliser des observations réelles afin de pouvoir confirmer ce résultat, ouvrant la voie à de futures améliorations dans les techniques d'optique adaptative et d'apodisation

% Je me repète complètement, c'est la même idée là mais j'aime bien les deux phrases quand meme :(

% En effet, ces planète étant généralement très proche de leurs étoiles (car leurs zone habitable sont de l’ordre de $$a=1\text{ UA}$$ comme la Terre en gros), et donc distinguer la lumière de la planète qui est 10^7 fois plus faible que celle de son étoile est un vrai challenge, au vu des distances angulaire à surmonter.

% %Ce qu'on sait déjà
% Une méthode pour surmonter ce problème est l’apodisation : on va venir créer une zone de haut contraste dans la lumière de l’étoile afin de permettre la récupération par l’instrument de la lumière de la planète, permettant de récupérer ces données et analyser la lumière réfléchis de cette planète pour analyser son atmosphere

% Les masques d’apodisations actuelles permettent d’observer jsuqu’a des 10-10 pour des planète assez éloigné et 10-5. Leurs paramètres sont (1.) IWA, (2.) OWA, (3.) T, et (4.) potentiel forme de la dark zone

% %Ce qu'on sait déjà - Limitation par le flou d'OA
% Nous verrons que bien que nous arrivons à atteindre de tes gros contraste, nous sommes limité par l’oa, qui vient créer un flou et limitant le contraste maximale atteignable par l’ELT-HARMONI, instrument sur lequel nous baserons nos recherche et simulation.

% %Méthodes
% À l’aide de simulation (1). réalisé dans l’approximation d’un flou d’optique adaptative uniforme dans notre champ de vue, et (2). d’une simulation plus réaliste avec de réels donnée de simulation d’OA, nous calculerons le gain en SNR des images de contrastes générés (PSF) pour une série d’apodiseur adéquat à l’observation de Proxima b.

% Pour ces deux situations, nous déterminerons le meilleur apodiseur

% Pour résumer, j’ai comparer des rapport de SNR d’apodiseur pour déterminer lequel était le meilleur pour l’observation de Proxima b

% - Il faut qu’il soit le plus grand possible (car $SNR^2$ = temps de téléscope en moins)

% - Il faut qu’il se trouve dans un certain intervalle de longueur d’onde

% - Il faut qu’on puisse récupérer toute l’information lumineuse de la planète pour toutes les longueurs d’onde d’observation

% %Résultats 
% Nous verrons qu’un masque optimal obtenu dans ce cadre de recherche correspond à un apodiseur de paramètres IWA = 2.6 OWA = 6.5 T = 0.70 sans bowtie 

% %Conclusion 
% On pourra donc gagner un certain temps de téléscope mais on est en définitive limité par un flou d’OA. Si on pouvait réduire le flou jusqu’à $10^(??)$, alors on pourrait espérer pouvoir augmenter le gain de SNR jusqu’à ???
\end{abstract}