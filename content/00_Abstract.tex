\begin{abstract}
%L'enjeu
% La question de notre solitude dans l'univers s'est posé tout au long de l'existence de l'humanité, 
\item \paragraph{Introduction}
À la question \textsl{"Sommes-nous seuls dans l'univers ?"}, l'analyse et la caractérisation d'atmosphère d'exoplanètes pourrait bien y répondre. La recherche de vie ailleurs étant l'une des plus grandes questions astronomiques actuelles, ces domaines jouent un rôle crucial dans la compréhension de notre singularité dans l'univers. Proxima b, l'exoplanète la plus proche de notre système solaire, est une cible de choix pour ces études/recherches.

\vspace{-1.9em}
\item \paragraph{Objectif} 
Les exoplanètes potentiellement habitables sont généralement situé très proche de leurs étoiles (\textbf{Source ?}), autour de 1 unité astronomique (\textbf{Source ?}), et donc distinguer la lumière de ces planètes depuis la Terre demande de devoir déciphérer deux lumières à séparation angulaire très faible.

\vspace{-1.9em}
\item \paragraph{Méthodes}
La méthode que nous utiliserons ici pour surmonter ce problème est \textbf{l'apodisation}. On va venir modifier la lumière de l'étoile pour créer une zone de haut contraste, permettant de récupérer la lumière de la planète. Cela permettra d'analyser la lumière réfléchie de cette planète pour en déduire sa composition atmosphérique.

\vspace{-1.9em}
\item \paragraph{Résultats}
On a donc trouvé que l'apodiseur le plus optimal pour l'observation de Proxima b est un apodiseur de paramètres IWA = 2.6 OWA = 6.5 T = 0.70 sans bowtie. Cela permettra de gagner un certain temps de télescope, mais on est en définitive limité par un flou d'OA. Si on pouvait réduire ce flou jusqu'à $10^{??}$, alors on pourrait espérer augmenter le gain de SNR jusqu'à ???

\vspace{-1.9em}
\item \paragraph{Conclusion}
On sera toujours limité par le flou, faudrai améliorer le flou mais c'est chaud patate

% Je me repète complètement, c'est la même idée là mais j'aime bien les deux phrases quand meme :(

% En effet, ces planète étant généralement très proche de leurs étoiles (car leurs zone habitable sont de l’ordre de $$a=1\text{ UA}$$ comme la Terre en gros), et donc distinguer la lumière de la planète qui est 10^7 fois plus faible que celle de son étoile est un vrai challenge, au vu des distances angulaire à surmonter.

% %Ce qu'on sait déjà
% Une méthode pour surmonter ce problème est l’apodisation : on va venir créer une zone de haut contraste dans la lumière de l’étoile afin de permettre la récupération par l’instrument de la lumière de la planète, permettant de récupérer ces données et analyser la lumière réfléchis de cette planète pour analyser son atmosphere

% Les masques d’apodisations actuelles permettent d’observer jsuqu’a des 10-10 pour des planète assez éloigné et 10-5. Leurs paramètres sont (1.) IWA, (2.) OWA, (3.) T, et (4.) potentiel forme de la dark zone

% %Ce qu'on sait déjà - Limitation par le flou d'OA
% Nous verrons que bien que nous arrivons à atteindre de tes gros contraste, nous sommes limité par l’oa, qui vient créer un flou et limitant le contraste maximale atteignable par l’ELT-HARMONI, instrument sur lequel nous baserons nos recherche et simulation.

% %Méthodes
% À l’aide de simulation (1). réalisé dans l’approximation d’un flou d’optique adaptative uniforme dans notre champ de vue, et (2). d’une simulation plus réaliste avec de réels donnée de simulation d’OA, nous calculerons le gain en SNR des images de contrastes générés (PSF) pour une série d’apodiseur adéquat à l’observation de Proxima b.

% Pour ces deux situations, nous déterminerons le meilleur apodiseur

% Pour résumer, j’ai comparer des rapport de SNR d’apodiseur pour déterminer lequel était le meilleur pour l’observation de Proxima b

% - Il faut qu’il soit le plus grand possible (car $SNR^2$ = temps de téléscope en moins)

% - Il faut qu’il se trouve dans un certain intervalle de longueur d’onde

% - Il faut qu’on puisse récupérer toute l’information lumineuse de la planète pour toutes les longueurs d’onde d’observation

% %Résultats 
% Nous verrons qu’un masque optimal obtenu dans ce cadre de recherche correspond à un apodiseur de paramètres IWA = 2.6 OWA = 6.5 T = 0.70 sans bowtie 

% %Conclusion 
% On pourra donc gagner un certain temps de téléscope mais on est en définitive limité par un flou d’OA. Si on pouvait réduire le flou jusqu’à $10^(??)$, alors on pourrait espérer pouvoir augmenter le gain de SNR jusqu’à ???
\end{abstract}