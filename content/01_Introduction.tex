%\section{Introduction}

\section{\centering Introduction - Proxima b et la recherche de vie ailleurs}
%Introduction ou contexte ? j'hésite
\setlength{\columnsep}{1.3em}%
\begin{wrapfigure}[15]{r}{0.45\textwidth}
    \includegraphics[width=0.45\textwidth]{figures/art_proxb.jpg}
    \caption{Vue d'artiste de Proxima Centauri b, l'exoplanète la plus proche de notre système solaire.}
\end{wrapfigure}

La recherche de planètes habitables a été identifié comme l'une des priorités majeures en astronomie selon le \textsl{Pathways to Discovery in Astronomy and Astrophyics for the 2020s} \cite{NAP26141}. En effet, la découverte de vie dans notre univers serait une avancée majeure pour l'humanité, et permettrait de répondre à des questions fondamentales sur notre place dans l'univers. 

Parmi les plus de 5000 exoplanètes découvertes à ce jour se trouve \textsl{Proxima b}, une exoplanète de type tellurique découverte en 2016, située dans la zone habitable de son étoile hôte Proxima Centauri. Étant donné sa proximité avec la Terre (à seulement 4.24 années-lumière), Proxima b est une cible de choix pour l'étude d'exoplanètes habitables. Non seulement son étude pourrait nous permettre de mieux comprendre les processus de formation et d'évolution des exoplanètes de type terrestres, mais la caractérisation de son atmosphère pourrait également nous permettre l'étude de potentielles biosignatures.
% Depuis la découverte de la première exoplanète en 1995, plus de 5000 exoplanètes ont été découvertes. Parmi elles se trouve Proxima Centauri b, une exoplanète de type tellurique située dans la zone habitable de son étoile hôte, Proxima Centauri. Cette exoplanète est donc une cible privilégiée pour la recherche de signes de vie. Cependant, l'observation directe de cette exoplanète est rendue difficile par la faible séparation angulaire entre l'étoile et l'exoplanète. Il est donc nécessaire de développer des techniques d'observation adaptées pour pouvoir caractériser cette exoplanète.\\ 

% En effet, ces planète étant généralement très proche de leurs étoiles (car leurs zone habitable sont de l’ordre de $$a=1\text{ UA}$$ comme la Terre en gros), et donc distinguer la lumière de la planète qui est 10^7 fois plus faible que celle de son étoile est un vrai challenge, au vu des distances angulaire à surmonter.

La possibilité de vie aussi proche de nous serait une découverte particulièrement marquante et symbolique pour l'humanité. Elle est donc une cible privilégiée pour la recherche de signes de vie. Cependant, son observation directe est rendue difficile par la faible séparation angulaire entre elle et son étoile. Située à une séparation angulaire de 37 mas, on pourrait comparer cette distance à la capacité de distinguer 2 objets à 6 mètres d'écart sur la Lune depuis la Terre. Il est donc nécessaire de développer des techniques d'observation adaptées pour pouvoir caractériser ce type d'exoplanètes proches de leurs étoile hôte.

Pour résoudre ce problème, l'ELT-HARMONI, un spectrographe de première lumière observant dans le visible et l'infrarouge (de 0.47 à 2.45 µm) prévu pour le télescope géant européen ELT, est équipé d'un Module Haut Contraste (HCM) développé à l'IPAG. Ce module permettra l'imagerie directe d'exoplanètes jusqu'à $10^{-6}$ fois plus faible que leur étoile hôte, et d'une séparation angulaire de 100 mas \cite{HCM} \cite{ZELDA}. Pour des raisons techniques lié au module, l'observation scientifique est limité dans la bande H, entre 1.45 et 1.8 µm.

Bien qu'il existe des outils permettant d'observer des contrastes de l'ordre de $10^{-10}$ pour des planètes assez éloignées et $10^{-5}$ pour des planètes proches, ils sont limités par le flou d'\textsl{optique adaptative}. Ce flou, provoqué par le mouvement des miroirs venant corriger l'impact de l'atmosphère sur la lumière, limite le contraste maximal atteignable par l'ELT-HARMONI.

À l’aide de simulation réalisé dans l’approximation d’un flou d’optique adaptative uniforme dans notre champ de vue (Partie ? du rapport), et d’une simulation plus réaliste à l'aide de données de simulation d’OA (Partie ?? du rapport), nous calculerons le gain en SNR des PSF générés pour une série d’apodiseur jugé potentiellement adéquat à l’observation de Proxima b.

% À l’aide de simulation (1). réalisé dans l’approximation d’un flou d’optique adaptative uniforme dans notre champ de vue, et (2). d’une simulation plus réaliste avec de réels donnée de simulation d’OA, nous calculerons le gain en SNR des images de contrastes générés (PSF) pour une série d’apodiseur adéquat à l’observation de Proxima b.

% Pour ces deux situations, nous déterminerons le meilleur apodiseur

% Pour résumer, j’ai comparer des rapport de SNR d’apodiseur pour déterminer lequel était le meilleur pour l’observation de Proxima b

% - Il faut qu’il soit le plus grand possible (car $SNR^2$ = temps de téléscope en moins)

% - Il faut qu’il se trouve dans un certain intervalle de longueur d’onde

% - Il faut qu’on puisse récupérer toute l’information lumineuse de la planète pour toutes les longueurs d’onde d’observation

Pour ces deux situations, nous déterminerons le meilleur apodiseur, défini par le plus grand gain en SNR atteignable; plus grand est le gain en SNR, plus grand est le gain de temps de télescope, proportionnel au carré du gain en SNR.

%Conclusion 
% On pourra donc gagner un certain temps de téléscope mais on est en définitive limité par un flou d’OA. Si on pouvait réduire le flou jusqu’à $10^(??)$, alors on pourrait espérer pouvoir augmenter le gain de SNR jusqu’à ???

