\section{Introduction - Proxima b}

\subsection{Contexte}


\subsection{Proxima et Proxima b, caractéristiques et intérêts d'observation}
% Résumé de l'article
% \begin{abstract}
% Ceci est le résumé de mon article.
% \end{abstract}

% Introduction
% \section*{Contexte}
% Depuis la découverte de la première exoplanète en 1995, plus de 5000 exoplanètes ont été découvertes. Parmi elles se trouve Proxima Centauri b, une exoplanète de type tellurique située dans la zone habitable de son étoile hôte, Proxima Centauri. Cette exoplanète est donc une cible privilégiée pour la recherche de signes de vie. Cependant, l'observation directe de cette exoplanète est rendue difficile par la faible séparation angulaire entre l'étoile et l'exoplanète. Il est donc nécessaire de développer des techniques d'observation adaptées pour pouvoir caractériser cette exoplanète.\\ 

% ELT-HARMONI est un spectrographe de première lumière observant dans le visible et l'infrarouge (de 0.47 à 2.45 µm) prévu pour le télescope géant européen ELT. Le Module Haut Contraste (HCM), développé à l'IPAG, permettra l'imagerie directe d'exoplanètes jusqu'à $10^{-6}$ fois plus faible que leur étoile hôte, et d'une séparation angulaire de 100 mas %\cite{SystemAnalysis} \cite{ZELDA}.

% \section*{Objectif}
% Pour ceci, on se propose de rechercher un masque de transmission adapté à l'observation de Proxima Centauri b. L'objectif est de trouver un masque de transmission qui permet de réduire le flux de l'étoile tout en conservant le flux de l'exoplanète.

% % Contenu de l'article
% \section{Section 1}
% Contenu de la section 1.

% %\Sfrac{1}{2}

% \section{Section 2}
% Contenu de la section 2.

% % Conclusion
% \section{Conclusion}
% Ceci est la conclusion de mon article.